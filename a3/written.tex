% Created 2019-08-10 Sat 22:31
% Intended LaTeX compiler: pdflatex
\documentclass[10pt]{article}
\usepackage[utf8]{inputenc}
\usepackage[T1]{fontenc}
\usepackage{graphicx}
\usepackage{grffile}
\usepackage{longtable}
\usepackage{wrapfig}
\usepackage{rotating}
\usepackage[normalem]{ulem}
\usepackage{amsmath}
\usepackage{textcomp}
\usepackage{amssymb}
\usepackage{capt-of}
\usepackage{hyperref}
\author{Jonathan Tow}
\date{\today}
\title{}
\hypersetup{
 pdfauthor={Jonathan Tow},
 pdftitle={},
 pdfkeywords={},
 pdfsubject={},
 pdfcreator={Emacs 26.2 (Org mode 9.1.9)}, 
 pdflang={English}}
\begin{document}

\textbf{Assignment 3 - Written}: Machine Learning \& Neural Networks


\section*{1. Machine Learning \& Neural Networks}
\label{sec:org73a526c}

\textbf{(a)} Adam Optimizer.

(i) Adam optimization uses a trick called \emph{momentum} by keeping track of \textbf{m}, a rolling average of the gradients:

\begin{LATEX}
\begin{align*}
\bold{m} &\leftarrow \beta_1 \bold{m} + (1-\beta_1) \nabla_{\theta} J_{minibatch}(\theta) \\
\theta &\leftarrow \theta - \alpha\bold{m}
\end{align*}
\end{LATEX}

where \(\beta_1\) is a hyperparameter between 0 and 1. Briefly explain how using \textbf{m} stops the update from varying 
as much and why this low variance may be helpful to learning, overall.\\


\textbf{Solution}: By keeping track of a rolling average of the loss function's gradients 
??????????????????????????????????????????????????????????????????????????????????
Dampening the oscillations is helpful because it straightens out the trail that descends into the minimum. 
Informally, the efficiency of optimization increases when update steps zig-zag (vary) less.

(ii) Adam also uses \emph{adaptive learning rates} by keeping track of \(\bold{v}\), a rolling average of the magnitudes of the gradients:

\begin{LATEX}
\begin{align*}
\bold{m} &\leftarrow \beta_1 \bold{m} + (1-\beta_1) \nabla_{\theta} J_{minibatch}(\theta) \\
\bold{v} &\leftarrow \beta_2 \bold{v} + (1-\beta_2) (\nabla_{\theta} J_{minibatch}(\theta) \odot \nabla_{\theta} J_{minibatch}(\theta))\\
\theta &\leftarrow \theta - \alpha \odot \bold{m} / \sqrt{\bold{v}}
\end{align*}
\end{LATEX}

Since Adam divides the update by \(\sqrt{\bold{v}}\) which of the model parameters will get larger updates? Why might this help 
learning?

\textbf{Solution:}

//

\textbf{(b)} Dropout is a regularization technique. During training, dropout randomly sets units in the hidden layer \(\bold{h}\) to
 zero with probability \(p_{drop}\) (dropping different units in each minibatch), and then multiplies \(\bold{h}\) by a constant
 \(\gamma\). We can write this as:

\[\bold{h}_{drop} = \gamma \bold{d} \circ \bold{h}\]

where \(\bold{d} \in {0, 1}^{D_k}\) is a mask vector.

(i) What must \(\gamma\) equal in terms of \(p_{drop}\)?\\

\textbf{Solution:}
??????????????????????????????????????????????????????????????????????????????????
Note that dropout treats each unit as a random variable so that the binary mask \(d\) follows a Bernoulli distribution
with distribution: \(p(X=1) = 1 - p_{drop}\) and \(p(X=0) = p_{drop}\). \(\gamma\) must be chosen such that the expected value
of \(\bold{h}_{drop}\) is \(\bold{h}\). 

(ii) Why should we apply dropout during training but not during evaluation?\\

\textbf{Solution:}
The goal of dropout is to reduce overfitting. We're interested in updating unit weights so as to form a network that
performs well across different datasets. Now, during evaluation we're concerned with how well the model handles unseen
data. When we dropout units, we're "thinning" out the network which in many cases will add noise to predictions and 
dampen accuracy. Thus, if we were to apply dropout during evaluation time, we would not be able to fairly assess the
generalization power of the network.

\newpage

\section*{2. Neural Transition-Based Dependency Parsing}
\label{sec:org083d954}

\textbf{(a)} \textbf{Transition-Based Parse}: A parser which incrementally builds up a parse one step at a time. At every step it 
maintains a \emph{partial parse} which is represented as:

\begin{itemize}
\item A \emph{stack} of words that are currently being processed.
\item A \emph{buffer} of words yet to be processed.
\item A list of \emph{dependencies} predicted by the parser.
\end{itemize}

Initially the stack contains ROOT, the dependencies list is empty, and the buffer contains all words of the sentence 
in order. At each step the parser applies a \emph{transition} to the partial parse until its buffer is empty and the stack 
size is 1. The following transitions can be applied:

\begin{itemize}
\item \emph{SHIFT}: removes the first word from the buffer and pushes it onto the stack.
\item \emph{LEFT-ARC}: marks the second (second most recently added) item on the stack as a dependent of the first item and 
removes the second item from the stack.
\item \emph{RIGHT-ARC}: marks the first (most recently added) item on the stack as a dependent of the second item and removes
the first item from the stack.\\
\end{itemize}

\textbf{Solution}:

\begin{longtable}{|l|l|l|l|}
Stack & Buffer & New Dependency & Transition\\
\hline
\endfirsthead
\multicolumn{4}{l}{Continued from previous page} \\
\hline

Stack & Buffer & New Dependency & Transition \\

\hline
\endhead
\hline\multicolumn{4}{r}{Continued on next page} \\
\endfoot
\endlastfoot
\hline
(ROOT) & [I, parsed, this, sentence, correctly] &  & Initial Config\\
(ROOT, I) & [parsed, this, sentence, correctly] &  & SHIFT\\
(ROOT, I, parsed) & [this, sentence, correctly] &  & SHIFT\\
(ROOT, parsed) & [this, sentence, correctly] & parsed->I & LEFT-ARC\\
(ROOT, parsed, this) & [sentence, correctly] &  & SHIFT\\
(ROOT, parsed, this, sentence) & [correctly] &  & SHIFT\\
(ROOT, parsed, sentence) & [correctly] & sentence->this & LEFT-ARC\\
(ROOT, parsed) & [correctly] & parsed->sentence & RIGHT-ARC\\
(ROOT, parsed, correctly) & [] &  & SHIFT\\
(ROOT, parsed) & [] & parsed->correctly & RIGHT-ARC\\
(ROOT) & [] & root->parsed & RIGHT-ARC\\
\end{longtable}

\textbf{(b)} How many steps will it take to parse \(n\) words (in terms of \(n\))?

\textbf{Solution}: In the worst case, parsing will take linear time, i.e. \(O(n)\). At any step of parsing, we have two possible
state transitions, either shifting a word from the buffer to the stack or clearing a dependent from the stack. Every 
word must spend a single step being shifted from the buffer, thus \(n\) words cost \(n\) shift steps. From the stack a 
word must be "arc"-ed over as a dependent exactly once, thus \(n\) words cost \(n\) "arc"-ing steps. Therefore, we have 
\(2*n\) steps giving a cost of \(O(n)\).


\textbf{(e)} Report of best UAS model:

\begin{longtable}{|c|c|}
dev UAS & test UAS\\
\hline
\endfirsthead
\multicolumn{2}{l}{Continued from previous page} \\
\hline

dev UAS & test UAS \\

\hline
\endhead
\hline\multicolumn{2}{r}{Continued on next page} \\
\endfoot
\endlastfoot
\hline
89.60 & 89.74\\
\end{longtable}

\textbf{(f)} For each sentence state the type of error, the incorrect dependency, and the correct dependency:\\

(i)\\

\begin{itemize}
\item \textbf{Error Type}:
\item \textbf{Incorrect Dependency}:
\item \textbf{Correct Dependency}:
\end{itemize}

(ii)\\

\begin{itemize}
\item \textbf{Error Type}:
\item \textbf{Incorrect Dependency}:
\item \textbf{Correct Dependency}:
\end{itemize}

(iii)\\

\begin{itemize}
\item \textbf{Error Type}:
\item \textbf{Incorrect Dependency}:
\item \textbf{Correct Dependency}:
\end{itemize}

(iv)\\

\begin{itemize}
\item \textbf{Error Type}:
\item \textbf{Incorrect Dependency}:
\item \textbf{Correct Dependency}:
\end{itemize}
\end{document}
